Here there are shown some scenarios to better understand the system usage from multiple viewpoints.

\paragraph{Scenario 1} \textbf{ User registration and log-in} \newline
Pierluigi is a Runner, he runs at least once per day and is very focused on the performance he gets during the training session.
In order to do so, he downloaded the Data4Help App  both on Smartwatch and mobile phone, so he registered, confirmed his email address, and then inserted personal data such as Age, Nationality, City of residence and the sport activity he practices.
Once having completed the subscription, he logs in through the Mobile app, turns on Bluetooth on smartphone and smartwatch and the Data4Help apps automatically synchronize in both devices. He sets “running activity” function on Mobile app and starts the run.



\paragraph{Scenario 2} \textbf{ User parameter consulting} \newline
Matteo uses Data4Help mobile app to track his sport activity. When he wants to see the old data from the Mobile app, he opens it, clicks on "Show past activity", inserts the date he's interested in, so the app shows all the activity data for that date from 0:00 to 24:00.


\paragraph{Scenario 3} \textbf{Data synchronization between devices} \newline
Letizia has just finished her Parkour lesson at Milano Gravity sports center.
She would like to see the calories she consumed, the maximum heart-bpm and other health parameters. Luckily, she has a Smartwatch with the Data4Help application installed. First, she turns on the Bluetooth on her smartphone. It automatically synchronizes the data with the smartwatch app, then she opens the Data4Help mobile app and gives a look on the recent statistics.



\paragraph{Scenario 4} \textbf{Hospital registration and purchase } \newline
Villa Serena is a hospital in Jesi trying to introduce an experimental way of monitoring its patients. In order to do so the director of the clinic goes to the website of the Data4Help services, subscribes using the email of the institution, clicks on “Subscribe to a Premium service” choosing the 1000-patients option. Then he adds a payment method and completes the operation.



\paragraph{Scenario 5} \textbf{Patient registration} \newline
Dr. Verdi is a private dermatologist in Milan who wants to use Data4Help in order to monitor his patients. He has already chosen to the  subscription plan with a limit of 100 patients. He asked Maria, one of his patients, to download the Mobile app and subscribe to the service.
Maria subscribes using her e-mail and fiscal code and then communicates her email to the Doctor who immediately adds Maria as her patient. Maria receives a notification asking if she agrees to be monitored by dr. Verdi (identified by his email) and she clicks on ‘accept’.



\paragraph{Scenario 6} \textbf{Company subscription and purchase} \newline
Clear-Water Spa is a company specializing on producing energy drinks for runners that wants to start a marketing campaign in Milan. In order to know where people usually do activities in Milan, it decides to subscribe to Data4Help services. The Marketing director goes to the Web page, subscribes to the service using e-mail and fiscal code, inserts a payment method and purchases the 1-City unlimited query option, specifying “Milan” as the city of preference.
Then he does his research on the city, inserting:
\begin{itemize}
    \item Age of the people to search (i.e. from 20 to 25 year old)
    \item The hour when to search people (i.e. from 9.00 to 10.00)
    \item The health parameter filters (i.e. heart rate $<$100 bpm$>$)
\end{itemize}
 The Data4Help system checks if the inquiry is satisfactory (i.e. whether it is too specific), if yes it shows a map with most frequent places of Milan where people go and an average of every health parameter, basing on the data provided by its users. 



\paragraph{Scenario 7}Subscription for the "Automated SOS" service \newline
Ottavio is a patient of Dr. Verdi using AutomatedSOS services offered by the Data4Help mobile app. The smartwatch of Ottavio has detected that his health parameters have gone below the threshold calculated by Data4Help, so the service is immediately notified and calls the ambulance of the closest hospital to the patient. The hospital receives the emergency notification  and sends an ambulance  to the position of Ottavio.


\paragraph{Scenario 8} Organizing a race \newline
Luigi is a race organizer for a famous sports brand in Milan. He wants to use the services offered by Track4Run in order to attract more runners, so first he goes to the Data4Help Mobile App and registers inserting his name, surname, email, Fiscal Code and specifying he is a Run organizer. After consolidating his email, he log in through the mobile app and clicks “Organize a new running event”. He fills in a module specifying the city of the race, the date, the starting time, the length in km and a description. Once finished, he clicks on “post the running event” and obtains a link where he can see all the details of the joining participants.



\paragraph{Scenario 9} A runner registers and starts the activity \newline
Salvatore is a professional runner of Milano.  He has recently been searching for some running events in order to test his performances, so he opens Data4Help app on his smartphone and clicks on “\textit{See the closest running event}”.
He is shown a map with all the races of Milan starting that day. Salvatore clicks on the closest one and a window appears on the screen of the app showing all the relevant data. Then he decides to join the race, so he clicks on “join the running event”. After that he receives the race number and a confirmation e-mail.



\paragraph{Scenario 10} Viewers of a race use the Data4Help mobile app \newline
Amanda and Dario have a son, Roberto, who has just joined a race organized through the Data4Help app. Since Roberto’s family wants to know their son geographical position during the race, they also download the app on their smartphones and register, specifying they are “\textit{spectators of a running event}”. Then they log in, click on “\textit{See nearby running events}”, choose the race they are interested in and click on “\textit{See runners}”. A map showing the position of all runners marked with blue points appears on the screen. The parents type the name of their son on the search box of the app and  his location becomes red and can be tracked.

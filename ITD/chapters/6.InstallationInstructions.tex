\subsection{Database setup}

\subsection{Backend setup}

\subsection{External services setup}
\subsubsection{Google Maps API}
In order to be able to use the Google Maps View in the Flutter project you need to obtain a Maps API key from Google. To do so, please refer to the following webpage:
\href{https://developers.google.com/maps/documentation/android-sdk/signup}{Get API Key | Maps SDK for Android}.

You should put the obtained key in the \texttt{AndroidManifest.xml} file. Please refer to the following document to get detailed instruction on how to do so:
\href{https://pub.dartlang.org/packages/google_maps_flutter}{google\_maps\_flutter | Flutter Packages}.

\subsection{Flutter Application setup}
In this section will be present all the necessary information to build the Flutter project.
\subsubsection{Requirements}
\begin{itemize}
    \item Internet connection
\end{itemize}

\subsubsection{Flutter}
In order to install the Flutter SDK with all its dependency, please refer to the following webpage: \href{https://flutter.io/docs/get-started/install}{Install - Flutter}.

Remember to also install all the needed dependencies to build Android Apps.

\subsubsection{Build the app for Android}
\begin{itemize}
    \item Download the project into a local folder;
    \item open a terminal and go into the project folder;
    \item launch the command \texttt{flutter doctor} to check that the project is ok;
    \item launch the command \texttt{flutter packages get};
    \item launch the command \texttt{flutter build apk} to actually build the apk.
\end{itemize}

\subsubsection{Known issues}
\begin{itemize}
    \item the app was developed under Android and has not been tested under iOS, so there can be some problem on this platform.
    \item You should include your personal Google Maps API as stated before, in order to make the Google Maps View work.
\end{itemize}
